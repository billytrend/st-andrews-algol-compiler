\documentclass{article}
\usepackage[utf8]{inputenc}

\title{Salgol Project - DOER}
\author{Billy Trend}
\date{September 2015}

\begin{document}

\maketitle

\section{Description}

Javascript is sometimes described as the 'assembly language of the web'. Many web applications are written in alternative languages but they all require compilation to Javascript. It is the common denominator of everything on the web. This means that the ubiquity of Javascript is unmatched by any other programming language: it is supported by almost all hardware and operating systems. The momentum behind Javascript makes it a reliable target for digital preservation. It is likely that there will always be some support for Javascript.

St Andrews Algol (Salgol) is a derivitave of ALGOL 60. It was developed in 1979 by Ron Morrison and Tony Davie. It was used as a teaching language for undergraduates at the university until 1999.

Since Salgol was developed on pre-UNIX operating systems, there are no compiler implementations that run on modern hardware. This leaves Salgol in danger of never being written again. This project is an effort to keep Salgol available for people to try out and even to develop web applications with it. It is also an exploration of the techniques for developing compile-to-Javascript languages.

\section{Objectives}

\begin{itemize}
\item Develop a compiler for Salgol to ES5 javascript.
\item Develop a JIT for Salgol using compiled Salgol.
\item Develop some parse tree optimisations.
\item Develop a gui for the compiler.
\item Write a report comparing the JIT, compiler approaches and efficiencies that were available.
\item Bonus: develop a compiler for Salgol to LLVM.
\item Bonus: develop a debugger for Salgol.
\item Bonus: enable DOM access and other side effects.
\end{itemize}

\section{Ethics}
None
\section{Resources}
\begin{itemize}
\item Open source libraries for convenience and user interface development.
\item Git/Github for version control.
\item Circle CI for continuous integration.

\end{itemize}

\end{document}

